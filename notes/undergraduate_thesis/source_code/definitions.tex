% Bibliography
\usepackage[
	hyperref=true,
	url=false,
	isbn=false,
	style=alphabetic,
	citereset=chapter,
	maxcitenames=3,
	maxbibnames=10,
	block=none]{biblatex}
\addbibresource{bibliography.bib}
\AtEveryBibitem{\clearfield{note}}

% Remove extraneous fields from bibliography.
\AtEveryBibitem{
	\clearfield{howpublished}
	\clearfield{pagetotal}
}

% Math
\usepackage{amsmath}
\usepackage{amsthm}
\usepackage{amssymb}
\usepackage{amsfonts}
\usepackage{dsfont}
\usepackage{mathrsfs}

\usepackage{tikz-cd}
\usepackage{tikz}
\usepackage{mathtools}

% Detecting if table is in math mode.
% Typesetting everything in table in math mode if so.
\usepackage{tabu}

% Colored links.
\usepackage{color}
\usepackage{xcolor}
\usepackage{hyperref}
\hypersetup{
	colorlinks,
	linkcolor={red!50!black},
	citecolor={blue!50!black},
	urlcolor={blue!80!black}
}
\usepackage{hyperref}
\usepackage{url}

\usepackage{graphicx}
\usepackage{subcaption}
\usepackage{enumerate}
\usepackage{wrapfig}
\usepackage [english]{babel}
\usepackage [autostyle, english = american]{csquotes}
\MakeOuterQuote{"}
\usepackage{epigraph}

% to make todo 
\usepackage{todonotes}

\graphicspath{{img/}}



% some theorem environments
% remove "[theorem]" if you do not want them to use the same number sequence
  \newtheorem{theorem}{Theorem}[chapter]
  \newtheorem{lemma}[theorem]{Lemma}
  \newtheorem{prop}[theorem]{Proposition}
  \newtheorem{cor}[theorem]{Corollary}

  \newtheorem{conj}[theorem]{Conjecture}

  \theoremstyle{definition}
  \newtheorem{defn}[theorem]{Definition}
  \newtheorem{conv}[theorem]{Convention}
  \newtheorem{ex}[theorem]{Example}
  \newtheorem{exs}[theorem]{Examples}
  \newtheorem{question}[theorem]{Question}
  \newtheorem{remark}[theorem]{Remark}
  \newtheorem{notn}[theorem]{Notation}

% esssup
\DeclareMathOperator*{\esssup}{ess\,sup}
\DeclareMathOperator*{\argmin}{arg\,min}
\DeclareMathOperator*{\cl}{cl}
\DeclareMathOperator*{\co}{co}
\DeclareMathOperator*{\dom}{dom}
\newcommand{\overbar}[1]{\mkern 1.5mu\overline{\mkern-1.5mu#1\mkern-1.5mu}\mkern 1.5mu}
\DeclareMathOperator*{\simd}{\stackrel{\text {d}}{\sim}}
\DeclareMathOperator*{\disteq}{\stackrel{\text {d}}{=}}
\DeclareMathOperator*{\distconv}{\stackrel{\text {d}}{\to}}
\DeclareMathOperator*{\probconv}{\stackrel{\text {Pr}}{\to}}
\DeclareMathOperator*{\asconv}{\stackrel{\text {a.s.}}{\rightarrow}}
\DeclareMathOperator{\erf}{erf}
\DeclareMathOperator{\erfc}{erfc}

\newcommand{\E}{\ensuremath{\mathbb{E}}}
\newcommand{\sumn}{\ensuremath{\sum_{i = 1}^n}}
\newcommand{\Ca}{\ensuremath{\operatorname{CVaR}_\alpha}}
\newcommand{\LL}{\ensuremath{\mathscr{L}}}
\newcommand{\cvar}{\ensuremath{\operatorname{CVaR}}}
\newcommand{\evar}{\ensuremath{\operatorname{EVaR}}}
\newcommand{\var}{\ensuremath{\operatorname{VaR}}}
\newcommand{\loss}{\ensuremath{\operatorname{Loss}}}


\DeclarePairedDelimiter\ceil{\lceil}{\rceil}
\DeclarePairedDelimiter\floor{\lfloor}{\rfloor}
